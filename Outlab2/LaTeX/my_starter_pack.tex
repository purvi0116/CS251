\documentclass[12pt, letterpaper]{article}
\usepackage{xcolor}
\usepackage[parfill]{parskip}
\usepackage{ragged2e}
\usepackage{hyperref}
%before \begin is the PREAMBLE
\begin{document}
%document environment
\title{LaTeX Starter Pack}
\author{190050087-190050090}
\date{August 30, 2020 }
\maketitle
\tableofcontents

\section{Introduction}
\label{intro}
\section{Basic Document Formatting}
\label{formatting}
%Page 2
\textbf{ Features we demonstrate}



\begin{itemize}
	\item Making a title and table of contents
	\item Organising the document into sections
	\item Setting up the page layout
	\item Designing our custom header and footer
	\item Formatting text
	\item Making (nested) lists
	\begin{enumerate}
		\item itemize
		\item enumerate
		\item description
	\end{enumerate}	
	\item Footnotes
	\item Typesetting mathematics
	\item Theorem and Proof environments
	\item Hyperlinks and cross references within the document 
	\item Custom environments
	\item Algorithms and code
	\item Inserting images
	\item Drawing tables
	\item Citations
\end{itemize}	



Descriptive lists are sometimes handy:


\begin{description}
	\item [CS 207] Discrete Structures
	\item [CS 213] Data Structures and Algorithms
	\item [CS 215] Data Analysis and Interpretation
	\item [CS 251] Software Systems Lab
	\item [CS 293] Data Structures and Algorithms Lab
\end{description}	
\section{Mathematics}
\label{maths}
\section{Computer Science}
\label{cs}
	\subsection{Algorithms}
	\subsection{Environments \& Code}
\section{Utilities}
\label{utils}
	\subsection{Images}
	\subsection{Tables}



%What is a quiet comment?
\end{document}